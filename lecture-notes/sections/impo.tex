%--
\chapter{不可能性结果}

%----------------------------------------
\section{概述}

当你熟悉分布式算法、分布式系统的基本概念,对它们有一点宏观性的思考的时候,一个很重要的概念就是分布式算法、分布式系统中的基本的不可能性结果。

\fur{早在89年的PODC上,Lynch就survey了“A hundred impossibility proofs for distributed computing”\cite{Lynch89},到了03年,Faith Ellen(与人合作)又写了“Hundreds of impossibility results for distributed computing”\cite{Fich03}。最终到了14年,Hagit Attiya和Faith Ellen汇集成书“Impossibility Results for Distributed Computing”\cite{Attiya14} \footnote{上述理论结果背后一个有趣的现象是,上面所提到的作者都是女性,分布式算法领域两本教科书的作者也是女性,她们的女性学生后续也在分布式理论领域做出了出色的结果。}。}

%----------------------------------------
\section{FLP}

不可能性结果的系统讨论超出introduction级别的教科书的范畴,有需要直接去看\cite{Attiya14}即可。但是有两组结果有一些特别的原因,专门提一下。

一个是异步容错共识的不可能性结果FLP。由于共识算法的大热,它也老被人提起。它的原始证明见\cite{Fischer85},直接看课本的表述更容易一些\cite{Attiya14}。

看这类证明,必须对分布式系统的执行有一个“超然世外的上帝视角”。分布式系统的执行就像是电影拍摄的片段、素材,你就像剪辑导演,随便复制、改写、拼接,直到得到满足你要求的执行。

%----------------------------------------
\section{拜占庭共识:“叛徒”比例不能达到$\frac{1}{3}$}

同步模型下解决拜占庭共识问题的一个基础结论是:当$3f>n$时,是不可能解决拜占庭共识问题的。其证明的关键在于,采用adversary argument的视角,可以让拜占庭错误的proc做你希望的任意动作,这样你就可以构造两个不可区分的执行。这两个执行的不可区分性与它们分别达成不同的共识,又是矛盾的。

具体的构造是巧妙的,需要结合\cite[Sec 5.2.3]{Attiya04}的图示来理解。证明是先讨论n=3的特殊情况,然后再据此推出任意n情况的证明。

%----------------------------------------
\section{Chain Argument}

这是不可能性结果证明中的一个简答而有效的技术,主要用于构造分布式系统执行之间的不可区分性(indistinguishability)。因为我们下面的理论证明用到了它,所以专门提一下。

这个技术可以让你很形象地理解什么叫做,以一个剪辑导演的视角,拼接分布式系统的执行片段。我们的PODC2020工作的会议报告中,有一个形象的简单例子 \footnote{https://www.bilibili.com/video/BV1K54y1D766/}。

具体内容可以看\cite[Chap 2]{Attiya14}。我自己由于下面研究的关系,实际看的是一份具体的研究工作\cite{Hadjistasi16}(这是一篇PODC16的短文,具体细节可以看它的arxiv长文版本)。

%----------------------------------------
\section{Fast Access to Atomic Registers}

实现atomic register经常需要要两轮(roundtrip)的通信,由此一轮通信的算法就被称为是fast的。

直觉上一轮通信是不可能实现atomic register抽象的,但是严格证明它却不那么容易。还是分single-writer和multi-writer的情况来逐个击破。

\cite{Dutta10}证明了single-writer情况的fast实现不可能。

Multi-writer情况的不可能性结果,期间有一些受限情况下的证明。我们自己的一份工作最终解决了这一问题\cite{Huang20}。

此外,既然fast是不可能严格实现atomic register的,那我们的另一系列的工作就讨论:假设fast是必须的,atomicity会被牺牲到什么程度。针对这一问题,我们提出了ASC(Almost Strong Consistency)的概念,并做了一些初步的理论分析与实验研究\cite{Wei17, Ouyang21}。
