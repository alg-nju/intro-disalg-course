%--
\chapter{类Paxos共识算法族}


包括Paxos \cite{Lamport01},Raft \cite{Ongaro14}和Zab \cite{Hunt10, Junqueira11}。

严格地讲,Zookeeper的Zab协议是一个atomic broadcast协议,由于atomic broadcast 和 consensus 两个问题可以互相规约,是等价的\cite{Chandra96unreliable},所以我们也经常不仔细区分,都称之为共识算法。

%--------------------------------------------------------------------------------
\section{Paxos} \label{}

%----------------------------------------
\subsection{基本Paxos算法}

%----------------------------------------
\subsection{Multi-Paxos}

%--------------------------------------------------------------------------------
\section{Zab} \label{}

\fur{我们对Zab算法,以及ZooKeeper的系统设计,进行了\tla 形式化规约。规约在Github开源\footnote{https://github.com/Disalg-ICS-NJU/zookeeper-tla-spec}。关于\tla 的简介参考附录\ref{Appen:TLA}。}

%--------------------------------------------------------------------------------
\section{Raft} \label{}


%--------------------------------------------------------------------------------
\section{复制状态机框架Rabia} \label{Sec_Rabia}

除了一些关系比较紧密的Paxos变体外,还有一些改动相对大一点的共识算法。它们仍然算是脱胎于Paxos算法,它们的设计也主要与Paxos的各种变体作充分对比。
所以这类算法仍然放在这一章。

专门提到Rabia的原因是,它的改进在方法学层面有比较重要的意义。

我们知道,共识算法的正确性要把safety和liveness分开来谈。\cite{Lamport01}中的基础Paxos算法,只能保证saftety,它把liveness分离开来单独谈。

由于FLP等一系列不可能性结果,异步环境的共识要保证liveness,必须要一些“额外的信息”。
常见的额外信息包括时间信息、随机性等。

此处的Rabia就是利用randomness,保证了livess。并且Rabia还很高效地保证了liveness。
一方面,用randomness保证liveness,省去了传统共识算法中leader选举,以及leader失效时选举新leader等复杂的机制。
另一方面,面对单数据中心场景中网络条件比较好的有利条件,Rabia还保证了良好的性能。

\fur{The Rabia SMR framework \cite{Pan21}.}