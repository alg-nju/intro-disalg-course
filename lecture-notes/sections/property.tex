%--
\chapter{分布式算法的基本性质}

%----------------------------------------
\section{Safety and Liveness}

本科生算法课中讨论的正确性规约,是必要的基础。基于此,我们可以发现这一规约对于讨论分布式系统做的对不对,是不够的。

分布式系统中的正确性讨论,主要基于safety和liveness这两个核心概念,它们是串行程序中“partial correctness”和“complete correctness”概念的泛化。

\fur{一直以为这两个核心概念是Lamport首创,\cite{Malkhi19}的Introduction章节中也有提及。但是后面又看到这两个概念是在\cite{Alpern85}中定义的(这一工作获得了2018年分布式计算理论领域的Dijkstra奖),Lamport应该是对于异步分布式算法,进一步深化了这两个概念。}

上述基本性质之外,还要讨论几个更深入的性质。

公平性(fairness)。

模拟(simulation)。从外在使用者的角度,它是计算模型之间的转换。从模拟算法设计者的角度,它是一种性质或者说规约。

下界与不可能性(impossibility)。这不是单个算法的性质,它是一个问题的所有可能算法的整体性质。

\fur{上述性质的讨论,主要参考了\cite[Sec 1.2]{Aspnes19}}

%----------------------------------------
\section{正确性规约的由来}

分布式系统设计的正确性规约,显然来自于用户的需求。但是除此之外,分布式系统的正确性规约还有一类更“微妙”的来源,这和分布式系统设计的基本范型(paradigm)有关。

分布式系统实际的核心挑战是,如何让独立的,只有局部知识的节点,共同完成一个全局性的任务。
应对这一挑战的主要手段,是设计一些“精巧”的全局不变式。不同节点间的协作,致力于维持这一全局不变式,而全局不变式的成立蕴含了用户需求的满足。

所以在很多时候,例如在分布式算法/系统验证的时候,我们所面对的正确性规约就是这样的全局不变式。具体的案例可以参考\S \ref{sec:zab}和\S \ref{sec:raft}。

