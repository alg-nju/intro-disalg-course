%--
\chapter{抽象计算模型的概念}

%----------------------------------------
\section{基本模型}

我们讨论抽象的分布式算法设计与分析,主要基于消息传递模型和共享内存模型。
这里将主要讲解计算模型的基本组成部分。而模型的具体细节,将放到具体的章节(第\ref{Chap_MSG}章、第\ref{Chap_SHM}章)去讲。

我们首先讨论计算模型的两个基本的维度。一个维度是通信的载体,一个维度是时间模型。

通信的载体主要包括消息传递(Message-passing,MSG)和共享内存(Shared Memory,SHM)。我们后续的章节也是按照这个维度来组织。

时间模型主要包括异步模型(asynchronous model)和同步模型(synchronous model),也包括更深入的半同步模型(partially synchronous model)。

\fur{参考两本教材中系统模型相关章节\cite{Aspnes19, Attiya04}。\cite[Sec 2.1.1]{Aspnes19},作者对两本书(\cite{Aspnes19} 和 \cite{Attiya04})中,对于“消息传递系统”不同建模方式的“异”和“同”进行了比较,并回顾了相关的历史。理解不同建模背后共性的部分,是学习如何对分布式系统进行合理抽象,并进行后续的问题定义、算法设计、算法分析的基础。
}


%----------------------------------------
\section{建模进阶}

建模中的一个重要概念是计算模型之间的模拟(simulation)。

\fur{Simulation的内容可以参考\cite[Part II]{Attiya04}。
    作者Attiya的代表工作\cite{Attiya95}是在MSG模型上,模拟一个SHM的atomic register。因为其研究的原因,Attiya在写教材的时候,也比较偏重模拟技术的讲解与应用。例如对于异步环境中共识不可能性的结论,她就专门使用模拟技术来证明。而直接证明的方式其实更有名,也非常有学习意义,它就是著名的FLP的证明\cite{Fischer85}。}

一个与“模型之间的模拟”有些类似,又有重要区别的概念是“问题之间的归约(reduction)”。在深入研究分布式计算的理论问题的时候,这两个技术是以既有结论为跳板,更“省劲”地证明新结论的利器。

其它还有一些高级的系统模型,可以作为主要模型的补充,对照着做一些初步的了解。

\fur{对于一些更高级模型的介绍,参考\cite[Part III]{Aspnes19}。}

%----------------------------------------
\section{编程抽象}

将系统建模推到一个更深层次的做法是,以“编程抽象”为核心载体,去解构一个分布式系统。对于分布式系统的使用者而言,分布式系统就是不同层级的各种编程抽象;对于分布式系统的构建者而言,其核心任务就是用设计层的算法、协议和系统层的代码去实现不同的编程抽象。在编程抽象的视角下,分布式系统就是层层编程抽象的组合,一个再复杂的系统,都像是乐高搭成的,都可以解构为更基本的单元。

这一编程抽象的视角,对于系统学习分布式系统理论而言,是非常有帮助的。对于实际开发者而言,它的主要意义在于将核心的概念,层层解构,辨析清楚。

\fur{\cite{Cachin11}这本书就是基于编程抽象,来组织分布式系统编程相关内容的。}