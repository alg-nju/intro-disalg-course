%--
\documentclass[UTF8]{ctexrep}

%-- layout
\usepackage[margin=1.0in]{geometry}

%-- url in bib
\usepackage{url}


%-- line space
\renewcommand{\baselinestretch}{1.5}

%--
\begin{document}

%--
\title{\bf \huge 分布式算法课程延伸阅读 \footnote{这是分布式算法课程推荐大家延伸阅读的文献。这个文献列表会在平时不断更新调整,包括增删条目、修改注释、调整文档的目录结构等。这些文献不是一个充分的覆盖,更像是一些经典文献的例举。}}

%--
\author{黄\ \ 宇 \\ http://cs.nju.edu.cn/yuhuang}

%--
\date{}
\maketitle

%--
\tableofcontents


%--------------------------------------------------------------------------------
\chapter{建模和模型}

\cite{Lamport78} 消息传递模型的基本概念。

\cite{Mattern89} 逻辑时钟。

\cite{Lamport86a, Lamport86b} 基础建模。包括atomic/regular/safe register的概念等。

\cite{Misra86} 硬件访问的公理化建模。

\cite{Chandra96} Failure detector的概念。

%--------------------------------------------------------------------------------
\chapter{Impossibility Results}

\cite{Attiya14} 不可能性结果的系统讲解书籍。

\cite{Fischer85} FLP结果。

%--------------------------------------------------------------------------------
\chapter{MSG算法}

\cite{Lamport01} Paxos算法。

\cite{Ongaro14} Raft算法。

%--------------------------------------------------------------------------------
\chapter{SHM算法}

\cite{Attiya95} 定义distributed atomic register。

\cite{Herlihy90} 定义linearizability。

\cite{Shao11} 多写register的regularity的定义。

\cite{Herlihy91} Consensus number的概念。

\cite{Herlihy11} Progress condition概念的系统阐述。

\cite{Lamport74} Bakery算法。


%--------------------------------------------------------------------------------
\chapter{经典系统}

\cite{Gilbert12} CAP 定理,由Eric Brewer在2000年提出,此处是12年后Gilbert和Lynch对CAP定理所作的阐释。

\cite{Abadi12} PACELC权衡,可以看做是CAP定理的拓展。

\cite{Li89} 分布共享内存系统。

\cite{Charron10} 涉及复制技术的经典理论、系统的系统阐述。

\cite{Chandra07} Google的Paxos实现。

\cite{Chang08} Google的BigTable。

\cite{Baker11} Google的Megastore。

\cite{Corbett12} Google的Spanner。

\cite{DeCandia07} Amazon的Dynamo。

\cite{Lakshman10} Facebook的Cassandra。

\cite{Hunt10} Yahoo的Zookeeper。

\cite{Burrows06} Google的chubby。

\cite{Gelernter85} 基于tupple space的协同。

%--------------------------------------------------------------------------------
\chapter{形式化规约与验证}

\cite{Burckhardt14-book} vis-ar框架:Replicated Data Type的declarative的规约框架。

\cite{Cerone18} vis-ar框架向分布事务isolation level的拓展。

\cite{Maric17} 共识算法的共性提炼与验证。

%--
\bibliographystyle{alpha}
%\bibliographystyle{apalike}
%--
\bibliography{disalg-reading}


\end{document}
