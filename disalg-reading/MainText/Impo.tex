%--
\chapter{Impossibility Results}

从道理上讲,“impossibility results”划出了理论的边界,我们才可以进而研究后续的算法设计与分析。这也是为什么impossibility results被放在了具体算法的前面。

但是实际上理论结果的形成,包括大家学习分布式算法的过程,却往往不是这样的。所以大家可以尽管跳过这一章,先了解后面的内容。待对分布式算法有了基本的认识之后,再回来深入研究经典的impossibility results。

%--
\begin{theorem}[\cite{Attiya14}]
    
     不可能性结果的系统讲解书籍。
    
\end{theorem}

%--
\begin{theorem}[\cite{Fischer85}]
    
    FLP结果。
    
\end{theorem}
