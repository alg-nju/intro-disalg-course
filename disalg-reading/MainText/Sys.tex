%--
\chapter{经典系统}

%--
\begin{theorem}[\cite{Gilbert12, Brewer12}]
    
    CAP 定理。
    
    Eric Brewer在2000年提出。
    
    \cite{Gilbert12}是12年后Gilbert和Lynch对CAP定理所作的阐释。
    
    \cite{Brewer12}是作者自己在12年后的思考。
    
\end{theorem}

%--
\begin{theorem}[\cite{Abadi12}]
    
    PACELC权衡,可以看作是CAP定理的拓展。
    
\end{theorem}

%--
\begin{theorem}[\cite{Li89}]
    
    分布共享内存系统。
    
\end{theorem}

%--
\begin{theorem}[\cite{Charron10}]
    
    涉及复制技术的经典理论、系统的系统阐述。
    
\end{theorem}

%--
\begin{theorem}[\cite{Chandra07}]
    
    Google的Paxos实现。
    
\end{theorem}

%--
\begin{theorem}[\cite{Chang08}]
    
    Google的BigTable。
    
\end{theorem}

%--
\begin{theorem}[\cite{Baker11}]
    
    Google的Megastore。
    
\end{theorem}

%--
\begin{theorem}[\cite{Corbett12}]
    
    Google的Spanner。
    
\end{theorem}

%--
\begin{theorem}[\cite{DeCandia07}]
    
    Amazon的Dynamo。
    
\end{theorem}

%--
\begin{theorem}[\cite{Lakshman10}]
    
    Facebook的Cassandra。
    
\end{theorem}

%--
\begin{theorem}[\cite{Hunt10}]
    
    Yahoo的Zookeeper。
    
\end{theorem}

%--
\begin{theorem}[\cite{Burrows06}]
    
    Google的chubby。
    
\end{theorem}

%--
\begin{theorem}[\cite{Gelernter85}]
    
    基于tupple space的协同。
    
\end{theorem}
