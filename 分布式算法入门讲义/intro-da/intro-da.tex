%--
\documentclass[UTF8]{ctexrep}


%-- layout
\usepackage[margin=1.2in]{geometry}

%--
%\usepackage{url, hyperref}


%-- user-defined commands
\newcommand{\myleaf}{\noindent \textleaf \ }

\newcommand{\code}[1]{$\textsf{#1}$}



%\linespread{1.1}

%--
\begin{document}


%--
\title{\Huge \bf 分布式算法入门}

%--
\author{黄\ \ 宇}

%--
\maketitle

%--
\tableofcontents

%--
\chapter*{前言}

每年招新学生的时候,分布式算法知识的入门是一个反复出现的任务。一般而言,分布式算法知识和大家的本科课程距离较远,难以自然地衔接。分布式算法的经典课本、论文在本科毕业的时候直接阅读往往比较困难。另外一方面,网上相关技术内容不少,但是质量堪忧,有误导性的“负作用”。为此陆续整理这一本讲义,希望能将每年重复给新学生罗列的学习内容沉淀下来,复用起来。

这本讲义完全是“on demand”式地逐步添加的,而且论述非常简要,更像是向学生讲解所需阅读的入门文献时,所做的注解。目前章节在tex源文件中的分布也是临时性的。未来有一天,如果这本讲义的内容充实到一定程度,或许它能成为一本合格的《分布式算法入门》课本。

%--
\part{基础知识}

%--
%--
\chapter{引言}

分布式系统的基础知识。

\myleaf 分布式系统的经典教科书 \cite{Tanenbaum06}。\\

抽象算法设计与分析的基础知识。

\myleaf 普遍使用的RAM模型。抽象的算法设计,归纳法证明正确性。抽象的算法分析,关键操作(critical operation)的计数 \cite{Cormen09, Huang20-book}。\\

形式化验证的基础知识

\myleaf 数理逻辑与软件形式化验证基础知识 \cite{Huth04}。\\

TLA+

\myleaf TLA+的基础知识 \footnote{Leslie Lamport, The TLA+ Home Page: http://lamport.azurewebsites.net/tla/tla.html}。

\myleaf TLA+教程 \cite{Wayne18}。\\

分布式系统的形式化验证的工业界实践。

\myleaf Amazon的实践 \cite{Newcombe15}。\\

面向教学的分布式算法快速实现。

\myleaf DSLabs \cite{Michael19}。

%--
\chapter{分布式计算模型}

%--------------------------------------------------------------------------------
\section{计算模型的基本维度}

%----------------------------------------
\subsection{一个基础的二维框架}

首先了解计算模型的基本概念。可以从两个维度来了解计算模型的基本构成。一个维度是时间,相应的有同步模型和异步模型。一个维度是(空间)通信方式,包括消息传递(简称为MSG)模型和共享存储(简称为SHM)模型。

\myleaf 阅读\cite{Attiya04}的第1、2、4章。阅读\cite{Aspnes19}的第2、15章。

\myleaf 计算模型的二维构成,参见L1胶片中的图。


%----------------------------------------
\subsection{失效模型的基本概念}

首先了解crash failure、Byzantine failure、link failure的基本定义和大致含义,后面结合具体的问题、算法、系统,做进一步了解。


%--------------------------------------------------------------------------------
\section{问题规约}

问题的规约(specification)。

安全性(safety)。

活性(liveness)。




%--------------------------------------------------------------------------------
\section{消息传递模型}

%----------------------------------------
\subsection{同步}

%----------------------------------------
\subsection{异步}


%--------------------------------------------------------------------------------
\section{共享内存模型}

能力强的原语\code{Test\&Set}。

能力弱的原语\code{Read}、\code{Write}。	


%--------------------------------------------------------------------------------
\section{计算模型基本概念的延伸与拓展}

%----------------------------------------
\subsection{模型间的模拟}


其次要了解计算模型之间的模拟(simulation)这一引申概念。了解模拟本身之外,还需要辨析、深入理解不同计算模型的优劣、权衡。

\myleaf 阅读\cite{Attiya04}的第7、9章。阅读\cite{Aspnes19}的第16章。


%----------------------------------------
\subsection{问题间的归约}

问题之间的归约(reduction)。

%--------------------------------------------------------------------------------
\section{特定领域分布式系统的建模}

%----------------------------------------
\subsection{RDMA、NVM等硬件对分布式系统经典模型的挑战}


Message-and-Memory模型。由于RDMA技术的出现,提出了更适合的计算模型,综合了MSG模型和SHM模型的特征。

NVM的出现,传统易失的内存和持久的磁盘之间距离的拉近。

\myleaf \cite{Aguilera19}。

%----------------------------------------
\subsection{分布式数据库系统}

数据划分(partition, sharding)的问题。

数据复制(replication)的问题。

对上层应用编程提供事务(transaction)的支持。

分布式数据库中的容错问题。


%----------------------------------------
\subsection{分布式消息分发中间件}

分布式消息分发系统,分布式消息中间件(DMS, Distributed Messaging System)。

发布-订阅(Pub-Sub, Publish-Subscribe)系统。


%--
\chapter{基本的分布式算法}

%----------------------------------------
\section{遍历}

广播。

分布式版的DFS。

%----------------------------------------
\section{领导者选举}

朴素Leader Election (LE),异步模型。

\myleaf \cite{Attiya04}相关章节。\\

%----------------------------------------
\section{分布共识}

朴素共识,同步模型。

\myleaf \cite{Attiya04}相关章节。


%----------------------------------------
\section{互斥}

基于\code{Test\&Set}原语的互斥算法。

\myleaf \cite{Attiya04}相关章节。 


%--
\part{消息传递}

%--
%--
\chapter{Paxos}

%--------------------------------------------------------------------------------
\section{Paxos算法的提出和Paxos算法族的演变}

Paxos提出的背景。

\myleaf 经典算法:经典的共识协议Paxos\cite{Lamport01}、VR\cite{Oki88}。

\myleaf 后续变体:ZK \cite{Junqueira11},Raft \cite{Ongaro14}。\\


 

%--
%--
\chapter{Raft}



%--
\chapter{原子性广播协议Zab}

Zab \cite{Junqueira11, Medeiros12}。


%--
\chapter{分布共享内存}

分布共享内存(Distributed Shared Memory, DSM)。



%--------------------------------------------------------------------------------
\section{原子寄存器}

%----------------------------------------
\subsection{单写}

%----------------------------------------
\subsection{多写}



%--
\chapter{分布事务的原子性提交}

原子性提交(atomic commitment)。 

%--
\part{共享内存}

%--
%--
\chapter{互斥}

Bakery算法

\myleaf \cite{Attiya04} 4.4节。\\ 

%--
\part{理论专题}

%--
%--
\chapter{不可区分性的构造}

经典的chain argument。

\myleaf 一个关于fast read-and-write atomic register的例子,参考\cite{Dutta10}。

更复杂、更精细的构造。

\myleaf 一个关于fast write atomic register的例子,参考\cite{Huang20}。 


%--
\part{系统专题}

%--
%--
\chapter{Paxos协议的系统实现}


%--
\chapter{Raft协议的系统实现}

etcd\footnote{https://etcd.io/}中的Raft实现。

BRaft\footnote{https://github.com/baidu/braft}。

MongoDB\footnote{https://www.mongodb.com/}中的Raft实现。


%--
\chapter{Zookeeper:Zab协议的系统实现} 

%--
%--
\chapter{容错与可靠性}

%----------------------------------------
\section{理解failure}

从节点的视角。crash failure。Byzantine failure。

\myleaf 经典的共识协议Paxos\cite{Lamport01}、VR\cite{Oki88},它们容忍的是crash failure。其中,VR扩展出了经典的拜占庭容错的共识协议PBFT \cite{Castro02}。\\


从网络连接的视角。

\myleaf \cite{Lynch96-textbook}第5章,容忍link failure的协同攻击问题。

\myleaf 大量真实但是anecdotal案例的精彩讲述 \cite{Bailis14-cacm}。基于大量开源项目bug report分析partial partition \cite{Alfatafta20}。

%----------------------------------------
\section{发现failure}

%----------------------------------------
\section{容忍failure}

错误自动注入技术。

\myleaf \cite{Alvaro18}。 

%--
\part{验证专题}

%--
%--
\chapter{分布式协议的模型检验}

主要以TLA+规约与模型检验为例。


%--
\chapter{分布式系统实现的模型检验}

\myleaf \cite{Guo11}。 

%--
%--
\chapter{分布式协议的定理证明}


%--
\chapter{分布式系统实现的定理证明} 


%-- bib
\bibliographystyle{alpha}
\bibliography{intro-da}


%--
\end{document}
